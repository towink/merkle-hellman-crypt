 \documentclass[12pt]{article}
\usepackage[utf8]{inputenc}
 
\usepackage[margin=1in]{geometry} 
\usepackage{amsmath,amsthm,amssymb}
\usepackage{algorithm}
\usepackage[noend]{algpseudocode}
 
\newcommand{\N}{\mathbb{N}}
\newcommand{\Z}{\mathbb{Z}}
 
\newenvironment{theorem}[2][Teorema]{\begin{trivlist}
\item[\hskip \labelsep {\bfseries #1}\hskip \labelsep {\bfseries #2.}]}{\end{trivlist}}

\newenvironment{definition}[2][Definición]{\begin{trivlist}
\item[\hskip \labelsep {\bfseries #1}]}{\end{trivlist}}

\newenvironment{example}[1][Ejemplo]{\begin{trivlist}
\item[\hskip \labelsep {\bfseries #1}]}{\end{trivlist}}
 
\begin{document}
 
%\renewcommand{\qedsymbol}{\filledbox}
 
\title{El criptosistema de Merkle-Hellman}
\author{Anna Pietrzak\\Diego Kiedanski\\Tobias Winkler\\Criptografía y Teoría de Códigos\\Universidad Complutense de Madrid} %if necessary, replace with your course title
 
\maketitle

\section{Introducción}


%\cite{shamir1984}
%\cite{merkle1978}

Cuando enviamos mensajes por cualquier canal de comunicación siempre nos exponemos al riesgo de que sean leídos por terceras personas antes de llegar a su destino. Supongamos que queremos mandar el mensaje
$$M = \text{COMPLUTENSE}$$
a nuestro compañero Bob. Para evitar que Eva, una tercera persona que de alguna manera ha conseguido interceptar $M$ en su camino, sea capaz de entender lo que le queremos decir a Bob, le podemos mandar una versión \emph{cifrada} de $M$, por ejemplo
$$M' = \text{DPNQMVUFOTF}.$$
Sin embargo, si hacemos eso tenemos que asegurarnos de que el destinatario Bob es capaz de \emph{descifrarlas} y sería ideal si él fuese la única persona con esa capacidad.

Para ese fin ciframos $M$ de tal manera que el resultado $M'$ dependa de una clave $K$. Más formal, podemos pensar en el cifrado como una función
$$e_K : \mathbb{M} \rightarrow \mathbb{M}'$$
que depende de $K$. Aquí $\mathbb{M}$ y $\mathbb{M}'$ serían los espacios de todos los mensajes y mensajes cifrados posibles, respectivamente. En el ejemplo anterior $e_K$ es la función definida por
$$e_K(m_1\ ...\ m_n) = m_1 + K\ ...\ m_n+K$$
donde la adición de una letra $m$ más un número entero $K$ significa avanzar $m$ por $K$ posiciones en el alfabeto (posiblemente volviendo de Z a A) y la clave que elegimos fue $K = 1$.

Solo las personas que disponen de $K$ serán capaces de entender, es decir descifrar, el mensaje original $M$. Igual que antes, podemos considerar este proceso como otra función
$$d_K : \mathbb{M}' \rightarrow \mathbb{M}$$
nuevamente dependiente de $K$. Entonces, en el ejemplo $d_K$ sería dado por
$$e_K(m_1\ ...\ m_n) = m_1 - K\ ...\ m_n - K.$$
¡Ahora solo nos tenemos que preocupar de que Bob sepa $K$ y ya estaremos listos para mandarle todos los mensajes que queramos!

\subsection*{Cifrado asimétrico}

¿Pero cómo nos podemos comunicar con Bob para intercambiar $K$? En realidad, eso es equivalente a enviar un mensaje con el contenido $K$ y si ese mensaje es interceptado, entonces la comunicación ya no es segura.

Para evitar este problema se han inventado los \emph{cifrados asimétricos}. Usan por lo menos dos claves, una clave \emph{secreta} $S$ y una clave \emph{pública} $P$. Para que le podamos enviar un mensaje a Bob necesitamos saber su clave pública $P$ y él va a tener que usar su clave privada $S$ para descifrarla. Con la notación de antes, el cifrado es una función $e_P$ dependiente de $P$ y el descifrado es $d_S$ que depende de $S$. El cifrado debe ser tal que sea muy difícil (en la práctica imposible) invertir $e_P$ solo conociendo la clave pública $P$.

\section{El cifrado de Merkle-Hellman}

Este criptosisteme es un ejemplo de cifrados asimétricos. Fue inventado por Merkle y Hellman en 1976 \cite{merkle1978}. Fue roto en 1984 por A. Shamir que presentó un algoritmo eficiente (polinomial) que es capaz de recuperar el texto cifrado solo a partir de la clave pública con alta probabilidad \cite{shamir1984}. Por lo tanto, este sistema \emph{ya no se debe usar en la práctica}. (??Más abajo veremos como se puede mejorar el sistema para recuperar la seguridad??) Antes de explicar como funciona tenemos que introducir un poco de teoría.

\subsection*{El problema de la mochila}

Una variante del \emph{problema de la mochila} es la siguiente cuestión: Dado una secuencia $M = m_1, ..., m_n$ de números enteros positivos (la \emph{mochila}) y un \emph{límite} $L$, ¿existen indices $I \subseteq \{1,...,n\}$ tal que $\sum_{i \in I}m_i = L$? Y si existen, ¿cuáles son? Esto también se conoce como el \emph{problema de la suma de subconjuntos}.

\begin{example}
Considera $M = 13, 1, 6, 3, 10$. Si $L = 20$, entonces el problema de la mochila relacionado tiene solución ya que $1+6+3+10 = 20$. Por otra parte, si $L = 8$ no tiene solución porque ninguna combinación de estos cinco números da $8$ en su suma.
\end{example}
%, el \emph{problema de la mochila} consiste en encontrar un subconjunto $S \subseteq M$ que maximice $R := \sum_{s \in S}s$ bajo la restricción que $R \leq L$.
%Un problema relacionado es el \emph{problema de la suma de subconjuntos}. Aquí la pregunta es: ¿Existe $S \subseteq M$ tal que $\sum_{s \in S}s = L$? Ese problema se puede considerar como un caso especial de él de la mochila y sigue siendo NP-completo. A continuación vamos a referirnos a este problema cuando decimos 'el problema de la mochila'.
Se ha demostrado que el problema de la mochila es NP-completo. El algoritmo más simple para resolverlo prueba todas las $2^n$ posible combinaciones de elementos de $M$. Para cada combinación el algoritmo tiene que sumar como máximo $n$ números. Sumar dos números de $r$ bit son se hace en $O(r)$ operaciones, y por lo tanto este algoritmo trivial está en $O(nr2^n)$ si los tamaños en bit de los números en $M$ están limitados por $r$. Como este algoritmo es exponencial en $n$, no se puede aplicar si $n$ es suficientemente grande. Existen mejores algoritmos que este pero el hecho de que el problema es NP-completo nos asegura que ninguno de ellos es polinomial (lo que no significa que no puede haber algoritmos que sean eficientes en muchos casos).

Como vamos a ver, la seguridad del cifrado se va a basar en la dificultad de resolver el problema de la mochila.

\subsection*{Mochilas supercrecientes}

En casos especiales, el problema de la mochila es fácil de resolver como vamos a ver enseguida.

\begin{definition}{1}
Sea $M = m_1, ..., m_n$ una secuencia ascendiente de números enteros positivos. Si $M$ verifica que
$$m_{i+1} \geq \sum_{k=1}^im_k$$
entonces $M$ se llama \emph{mochila supercreciente (MS)}.
\end{definition}
\textbf{Ejemplo}
\begin{itemize}
\item
$M := 3, 4, 11, 42$ es una MS porque $4 > 3$, $11 > 3 + 4 = 7$ y $42 > 3 + 4 + 11 = 18$.

\item 
Para $n \in \N$, $M := 2^0, 2^1, ..., 2^n$ también es una mochila supercreciente ya que todo $i \in \N$ verifica que
$$2^{i+1} - 1 = \sum_{k=0}^i2^k.$$
\end{itemize}
La secuencia del último ejemplo es la 'menor' MS posible, es decir todos los elementos $m_i$ de cualquier MS verifican que $m_i \geq 2^{i-1}$. Esto implica que los elementos de todas las MS de longitud $n$ se representan con por lo menos $n-1$ bits y su suma tiene al menos $n$ bits.

Observamos que si $M = m_1, ..., m_n$ es supercreciente y $m_n \leq L$, entonces si $M$ tiene una solución, es necesario que $m_n$ esté en la solución porque si no, no podremos alcanzar $L$ ya que $m_n$ es mayor que la suma de todos los demás elementos de $M$.
De esta observación podemos concluir el siguiente algoritmo:
\vspace{1em}
\begin{algorithmic}[1]
\Procedure{ResolverMS}{$m_1,...,m_n$, $L$}
\State $sol \gets \emptyset$
\For{$i = n,...,1$}
	\If{$m_i \leq L$}
		\State $sol \gets sol \cup \{m_i\}$
		\State $L \gets L - m_i$
	\EndIf
\EndFor
\State \textbf{return} $sol$
\EndProcedure
\end{algorithmic}
\vspace{1em}
El algoritmo asuma que $M = m_1,...,m_n$ es una mochila supercreciente y que los $m_i$ están en orden ascendiente. Su complejidad de tiempo es $O(n)$ (lineal en $n$) porque consiste de un solo bucle de exactamente $n$ iteraciones.

\subsection*{Cifrar y la clave pública}

Para cifrar un bloque $B = b_1...b_n$ de $n$ bits tomamos una mochila supercreciente $M = m_1, ..., m_n$ de longitud $n$. Entonces eligimos dos números $q, r \in \Z$ tal que $q > \sum_{i=1}^nm_i$, $r > 1$ y el máximo común divisor de $q$ y $r$ sea $1$. Llamamos \emph{modulo} a $q$ y \emph{multiplicador} a $r$. Se sugerieron valores alrededor de $n = 100$ y $q$ de $200$ bits. Calculamos
$$M' = \{m_1', ..., m_n'\} = \{r m_1 \text{ mod } q, ..., r m_n \mod q\}.$$
Es importante entender que $M'$ en general ya no es supercreciente debido a las operaciones de modulo. Obtenemos el bloque cifrado $B'$ sumando aquellos elementos de $M'$ cuyos indices corresponden a los bits que valen uno en nuestro bloque $B$, es decir hacemos la suma
$$B' = \sum_{i=1}^nb_im_i'.$$
$B'$ es un solo número en $\Z$.

¿Cómo obtenemos una mochila supercreciente? Se puede generar una MS aleatoriamente de la siguiente manera: Se elige un parametro $\texttt{salto}\ > 1$. Sean $a_1, ..., a_n$ elementos tomados de la distribución uniforme de los números enteros de $1$ a \texttt{salto}. Definimos $m_1 := a_1$ y $m_{i+1} := a_{i+1} + \sum_k^n m_{k}$. Entonces $m_i \leq \texttt{salto} \cdot 2^{i-1}$ por inducción:
\begin{itemize}
	\item $i = 1$: $m_1 \leq \texttt{salto}$ es correcto
	\item $i > 1$:
	$$m_{i} = a_i + \sum_{k=1}^{i-1} m_{k} \leq a_i + \texttt{salto} \cdot \sum_{k=1}^{i-1}2^{k-1} \leq \texttt{salto} + \texttt{salto} \cdot ( 2^{i-1} - 1) = \texttt{salto} \cdot 2^{i-1}$$
\end{itemize}
Entonces, el tamanio en bit de los $m_i$ no es más de $log_2(\texttt{salto}) + (i-1)$.

\begin{example}
Supongamos que queremos cifrar el número 157. Su representación binaria es $B = 10011101$. Entonces, como explicado antes, tenemos que eligir una mochila supercreciente de longitud $8$, por ejemplo
$$M = \{1, 3, 6, 11, 27, 53, 111, 213\}$$
y los números $r$ y $q$. Como la suma de todos los elementos en $M$ es igual a $425$, podemos tomar $q = 499$ y $r = 101$. Como $101$ es primo, $mcd(q,r) = 1$. La mochila $M'$ resultante sería
$$M' = \{101, 303, 107, 113, 232, 363, 233, 56\}$$
que claramente no es supercreciente. Ya podemos calcular el bloque cifrado:
$$B' = 101 + 113 + 232 + 363 + 56 = 865$$
\end{example}
Para cifrar solo se necesita la mochila transformada $M'$, por eso la clave pública es $K_P = M'$.

\subsection*{Descifrar y la clave secreta}
Como da igual si primero sumamos y después tomamos modulo o al revés, la siguiente equivalencia es correcta:
$$B' = \sum_{i=1}^nb_i(rm_i \mod q) \equiv r\big(\sum_{i=1}^nb_im_i\big)\mod q$$
y por lo tanto
$$\sum_{i=1}^nb_im_i \equiv r^{-1} \cdot B' \mod q.$$
$r^{-1}$ existe porque $mcd(r,q) = 1$. Ahora para averiguar los $b_i$ solo tenemos que resolver el problema de mochila con $M$ supercreciente y eso se puede hacer muy rápido como hemos visto antes.

\begin{example}
Tenemos $r^{-1} = 101^{-1}= 84 \mod q$ lo que se puede calcular eficientemente con el algoritmo extendido de Euclides. Entonces,
$$r^{-1} \cdot B' = 84 \cdot 865 \equiv 305 \mod 499.$$
Lo único que falta es llamar el algoritmo \textsc{ResolverMS}($M, 305$) que nos da el resultado $305 = 1 + 11 + 27 + 53 + 213$. Estos números de la mochila $M$ corresponden precisamente al bloque original $10011101$.
\end{example}
Para descifrar se necesita $r^{-1}, q$ y la mochila original $M$, por lo tanto tenemos la clave secreta $K_S = (M, r^{-1}, q)$.

\section{Implementación}

Hemos implementado el sistema de Merkle-Hellman en Python. Existen los scripts \textbf{key\_generator}, \textbf{encrypt} y \textbf{decrypt} que se utilizan como se explica a continuación:

\begin{itemize}
	\item \textbf{python key\_generator.py \textit{n} \textit{jump}}: Crea dos archivos \textit{clave.priv} y \textit{clave.pub} que forman una pareja de claves. Se van a generar claves (mochilas) de longitud $n$ y saltos entre $n$ y \textit{jump} como se ha explicado anteriormente.
	\item \textbf{python encrypt.py \textit{key} \textit{message}}: Genera el archivo \textit{message.crp} que se obtiene cifrando \textit{message} con la clave pública \textit{key}.
	\item \textbf{python decrypt.py \textit{key} \textit{message}}: Genera el archivo \textit{message.dcrp} que es el resultado de descifrar el mensaje \textit{message} con la clave secreta \textit{key}.
\end{itemize}
Aparte de esto también creamos el script \textbf{bruteforce} que, dado un mensaje cifrado y la clave pública, intenta reconstruir el mensaje original resolviendo el problema de la mochila \emph{no supercreciente} tras probar todas las posibles combinaciones. Este programa solo sirve para evaluar la seguridad y no forma parte del criptosistema.


\bibliographystyle{alpha}
\bibliography{references}

 
\end{document}